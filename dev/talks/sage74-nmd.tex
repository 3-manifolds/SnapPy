% "aspect" options are 4:3, 16:10, 16:9 Option "half" cuts the width
% in half for later 2up-ing. 

% As of August 2015, projectors in UIUC math dept seem to be Epson
% G5650W's with a 16:10 aspect ratio and a resolution of 1280x800

\documentclass[aspect=16:10, tikz, half]{nmd/slide}   

\begin{document}
\begin{frame}
  \emph{\Large SnapPy:Part I}

  Talk at Sage Days 74  \\
  Observatoire de Paris, May 2016

  \vspace{1.5cm}
  SnapPy developed by  Marc Culler, Nathan Dunfield, Matthias Goerner,
  and Jeff Weeks.  
  \vspace{0.5cm}

  With contributions from Mark Bell, Tracy Hall, Saul Schleimer, Malik
  Obeidin, Robert Lipschitz, Jennet Dickinson and many others...
 
\end{frame}


\begin{frame}
Thurston/Perelman: \\ 
``Topology $=$ geometry in dim 3''

\vspace{1cm}

SnapPy's topological objects:
\begin{itemize}
  \item Compact $M^3$ with $\partial M$ tori.  
  \item Knots/links in $S^3$ via (spherogram and plink).
  \item Bundles/Heegaard splittings (twister by [BHS]).
\end{itemize}
 
\end{frame}

\begin{frame}
  SnapPy's geometric focus: complete Hyperbolic structures on
  $\mathrm{int}(M)$.  

  \begin{itemize}
    \item Volume, Chern-Simons, length spectrum, ...
    \item Dirichlet domains, horoball packings, ...
    \item Practical solution to the homeomorphism problem for
      hyperbolic 3-manifolds. 
  \end{itemize}
\end{frame}


\begin{frame}
  Under the hood, SnapPy is:
  \begin{itemize}
  \item C kernel: 40K LOC started by Weeks in 1990. 
  \item Cython wrapper: 5K LOC.
  \item Python code: 20K LOC.
  \item Databases with 2M+ manifolds.
  \item Not SageMath specific, works on Linux, OS X, and Windows.
  \item Modules hosted on PyPI.
  \item Has a Tk-based GUI with OpenGL-based 3D graphics.
  \item User-friendly native installers provided for OS X and Windows.    
  \end{itemize}
\emph{Why we're here: SnapPy and SageMath are friends!}
\end{frame}


\begin{frame}
  Live demo 1:
  \begin{itemize}
    \item Start stand-alone app. Point out IPython window.  
    \item Draw knot using Plink, send to SnapPy.
    \item Use browse, show off 3D graphics and identify.
  \end{itemize}
\end{frame}


\begin{frame}
  Core topological data structure is a \emph{Triangulation}: an ideal
  triangulation $\cT$ of $M^3$ with $\partial M$ a \emph{nonempty}
  union of tori, possibly with Dehn fillings.
  
  \bigskip

  $\cT$ is a CW complex obtained from a union of tetrahedra by gluing
  faces in pairs by affine maps, such that $\cT \setminus
  \mathrm{int}(N(\cT^{0})) \cong M$. 

  \bigskip

  Each component $T$ of $\partial M$ has a framing, i.e.~a choice of basis
  for $H_1(T; \Z)$. 
\end{frame}

\begin{frame}
Finding hyperbolic structures on $\cT$.
\begin{itemize}
\item Shapes of hyperbolic tetrahedra in $\H^3$.
\item Gluing and completeness equations.
\item Numerically solve using Newton's method: double, quad-double,
  and arbitrary precision. 
\item No guarantees but works extraordinarily well in practice, as
  seen in this plot of volumes of random knots.
\end{itemize}
\end{frame}

\begin{frame}
  SnapPy has extra features in SageMath: 
  \begin{itemize}
    \item Using GAP/Magma to find large finite covers.
    \item Hyperbolically twisted Alexander polynomials.
    \item Computing associated number fields. 
    \item Verified computation.
    \item Spherogram links to SageMath braids (soon!).
  \end{itemize}
\end{frame}


\begin{frame}
  Live demo 2:
  \begin{itemize}
    \item Sage-specific stuff, I guess??
  \end{itemize}
\end{frame}

\end{document}