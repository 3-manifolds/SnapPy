\documentclass[tikz, a4paper]{nmd/hw}
\begin{document}
\classtemplate{Nathan PS}{Warwick School}{\hfill}

For further references and links: \url{http://dunfield.info/warwick2017}

\begin{problems}
\item Put a hyperbolic structure
  \[
    \Sigma = S^2 - \{\mbox{three points}\}
  \]
  as follows.  Fix an ideal triangle $T$ in $\H^2$ and double it
  across the three boundary geodesics.  That is, make two copies of
  it, call them $T_1$ and $T_2$, and glue the sides together by the
  ``identity map''.  Prove that the hyperbolic metric on $\Sigma$ is
  complete.

\item Suppose $\cT$ is a (topological) ideal triangulation of
  punctured surface $\Sigma$.  As per the lecture, we can put a
  hyperbolic structure on $\Sigma$ by assigning a ``shear'' to each
  edge.
  \begin{enumerate}
    \item Formulate conditions on the shears that are equivalent to
      the hyperbolic structure being complete.
    \item Use your answer in (a) and an Euler characteristic
      calculation to prove that the dimension of the
      Teichm\"uller space of a surface of genus $g$ with $k$ punctures
      has (real) dimension $6g - 6 + 2k$.
    \end{enumerate}

  \item Consider the upper halfspace model for $\H^3$.  The plane $E$
    at height $1$ has the standard Euclidean metric; it is an example
    of a \emph{horosphere}.  Given a compact
    region $R$ in $E$, consider the ``chimney'' over it:
    \[
      C(R) = \setdef{ (x, y, t) \in \H^3}{\mbox{$(x, y, 1) \in R$ and
          $t \geq 1$} }
    \]
    Prove that $C(R)$ has finite volume and relate its volume to the
    area of $R$.

  \item Consider the a geodesic ideal tetrahedron $T$ in
    $\H^3$. Recall from Lecture 2 that each edge of $T$ has an
    associated \emph{shape parameter}.
    \begin{enumerate}
    \item Prove that the shape parameter does not depend on an
      orientation of the edge and that opposite edges have the same
      shape parameter.

      Hint: Find symmetries of $T$ by looking at the perpendicular
      bisectors between pairs of opposite edges.

    \item Prove the formulas that relate the different edge parameters
      given in lecture. 
    \end{enumerate}

  \item Figure out how to get SnapPy to give you the edge equations of
    an ideal triangulation.  
    
\end{problems}


\end{document}