\documentclass[tikz, a4paper]{nmd/hw}
\begin{document}
\classtemplate{SnapPy worksheet}{CURVES 2015}{\hfill}

For links for SnapPy, SageMath, and other needed topics see: \url{http://dunfield.info/curve}.

\begin{problems}
  \item Programming and searching.
    \begin{enumerate}
      \item Find all manifolds in the \texttt{OrientableClosedCensus}
        which have a 2- or 3-fold covering space with $b_1 > 0$.  
      \item The smallest volume manifold in the
        \texttt{OrientableClosedCensus} with $b_1 > 0$ is $m160(3,1)$
        which has volume $\approx 3.1663333212$.  Find a closed
        manifold with $b_1 > 0$ with smaller volume by searching
        through 0 surgeries on the knots in \texttt{HTLinkExteriors}.  

      \item Express the first manifold $M$ that you found in (b) as a
        Dehn filling on a 1-cusped manifold in the
        \texttt{OrientableCuspedCensus}.  

      \item The reason that the closed manifold $M$ in part (c) is not
        in the \texttt{OrientableClosedCensus} is that its shortest
        geodesic has length smaller than the cutoff that was chosen by
        Hodgson and Weeks when they created this census.  Determine
        this cutoff by finding length of the shortest geodesic for the
        all the manifolds in the \texttt{OrientableClosedCensus}.
    \end{enumerate}

\item Recall that two 3-manifolds are commensurable if they have a
  common finite cover.  
  \begin{enumerate}
  \item In the \texttt{OrientableClosedCensus}, find several pairs of
    1-cusped manifolds that appear to have the same volume. (You don't
    have to find all such pairs, though that can certainly be done.)

  \item For each pair, increase your confidence that the volumes are
    the same by using the \texttt{ManifoldHP} type, which works with
    quad-double precision.

    \item For each pair, try to either (a) find a common finite cover,
      or (b) show they are incommensurable by looking at the cusp
      density.  
  \end{enumerate}

\item SageMath and SnapPy are friends.
  \begin{enumerate}
  \item Find the first nontrivial knot in \texttt{HTLinkExteriors}
    which has the same Alexander polynomial as the unknot.

  \item Let $M$ be the knot exterior you found in part (a).  Via
    Matthias's new verify module in SnapPy, use interval arithmetic to
    rigorously prove that this manifold is indeed hyperbolic.
    Consequently, the corresponding knot is not the unknot. 

  \item Find exact expressions for the tetrahedra shapes of
    $M$, which live in some number field.  

  \item Check that the number field found in (c) is the same as the
    trace field of $M$.  
  \end{enumerate}

\item SageMath and SnapPy are friends, part II.
  \begin{enumerate}
    \item Find many cusped hyperbolic 3-manifolds in the various
      censuses whose trace fields are $\Q(i)$.  Hint: Computing trace
      fields is really expensive, so either (a) try with very small
      parameters or (b) numerically recognize elements of $\Q(i)$ that
      have small denominators.

    \item Amongst the your examples in $(a)$, find one with a 
      non-integral trace.  This manifold will not be arithmetic and
      will also contain a closed incompressible surface.  
  \end{enumerate}

\end{problems}


\end{document}