\documentclass[tikz, a4paper]{nmd/hw}
\begin{document}
\classtemplate{Nathan PS}{Warwick School}{\hfill}

For further references and links: \url{http://dunfield.info/warwick2017}


\begin{problems}
\item Suppose $N$ is a compact orientable 3-manifold with $\partial N$
  a nonempty union of tori.  Let $M = N \setminus \partial N$ be its
  interior.  Show that for any ideal triangulation $\cT$ of $M$ the
  number of tetrahedra in $\cT$ is equal to the number of edges.

\item Prove that any ideal tetrahedra in $\H^3$ has volume less than
  1000. (In fact, the volume is less than
  $1.01494160640965362502121$.)

\item As mentioned in Lecture 2, the complement $M$ of the figure 8
  knot fibers over the circle with fiber a once-punctured torus $T$.
  The monodromy of this bundle is given by $\mysmallmatrix{1}{1}{1}{2}
  \in \SL{2}{\Z} \cong \Out(F_2)$ which acts on the free group
  $\pi_1(T) = \pair{\alpha, \beta}$ by
  \[
    \alpha \mapsto \alpha \beta \qquad \beta \mapsto \beta \alpha \beta
  \]
  so that
  \[
    \pi_1(M) = \spandef{\alpha, \beta, \mu}{\mu \alpha \mu^{-1} =
      \alpha \beta, \ \mu \beta \mu^{-1} =  \beta \alpha \beta}
  \]
  Is this element of $\Out(F_2)$ fully irreducible (iwip)?

  \item As mentioned in Lecture 3, every cusped hyperbolic 3-manifold
    has a ``canonical cell decomposition'' which is typically a
    triangulation and hence often referred to as the ``canonical
    triangulation''.
    \begin{enumerate}
      \item Figure out the method in SnapPy for replacing a
        triangulation with the canonical one.  Use this to find an
        example where the the canonical triangulation does \emph{not}
        minimize the number of tetrahedra.

      \item If you ``browse'' a manifold in SnapPy, the ``Cusp Nbhds''
        tab shows what you see if you stand infinity in the first
        cusp of the 3-manifold and look inside.  It includes a
        visualization of both the canonical triangulation and its
        dual, called the Ford domain.  Use the ``View options'' menu
        to turn on and off different parts of the visualization and
        try to understand what it is the picture is telling you.
    \end{enumerate}
  
  \item Here's how you get the exterior of a randomly chosen 14-crossing prime knot:

  \texttt{knots = HTLinkExteriors(cusps=1, crossings=14)} \\ 
  \texttt{M = knots.random()}

  \begin{enumerate}
    \item Python uses the \texttt{len} function to access the length of any
      list-like object, so do \texttt{len(knots)} to see how many such
      knots there are.

    \item Try creating the Dirichlet domain for $M$ at the command
      line.  Most of the time you will get an error message saying
      that this failed!  (If not, pick a different example which does
      fail for the rest of this problem ;-).

    \item By default, the hyperbolic structure on $M$ is computed
      using standard double-precision floating-point numbers (about 15
      decimal digits).  It turns out this isn't enough to find the
      Dirichlet domain for a manifold this complicated.  Use the
      \texttt{high\_precision} method of $M$ to upgrade it to a
      \texttt{ManifoldHP} called $H$. 

    \item Compute the volumes of $M$ and $H$.  Are the answers
      consistent with the hyperbolic structure on $H$ being computed
      to quad-double precision?

    \item Try computing the Dirichlet domain $D$ using $H$, which
      will most likely succeed though it typically takes a few
      seconds. 

    \item Interrogate $D$ to get a pretty picture and find out how
      many faces and vertices $D$ has. 

    \end{enumerate}
    
  
  \item Programming and searching.
    \begin{enumerate}
      \item Find all manifolds in the \texttt{OrientableClosedCensus}
        which have a 2- or 3-fold covering space with $b_1 > 0$.  
      \item The smallest volume manifold in the
        \texttt{OrientableClosedCensus} with $b_1 > 0$ is $m160(3,1)$
        which has volume $\approx 3.1663333212$.  Find a closed
        manifold with $b_1 > 0$ with smaller volume by searching
        through 0 surgeries on the knots in \texttt{HTLinkExteriors}.  

      \item Express the first manifold $M$ that you found in (b) as a
        Dehn filling on a 1-cusped manifold in the
        \texttt{OrientableCuspedCensus}.  

      \item The reason that the closed manifold $M$ in part (c) is not
        in the \texttt{OrientableClosedCensus} is that its shortest
        geodesic has length smaller than the cutoff that was chosen by
        Hodgson and Weeks when they created this census.  Determine
        this cutoff by finding length of the shortest geodesic for the
        all the manifolds in the \texttt{OrientableClosedCensus}.
    \end{enumerate}

\item Recall that two 3-manifolds are commensurable if they have a
  common finite cover.  
  \begin{enumerate}
  \item In the \texttt{OrientableClosedCensus}, find several pairs of
    1-cusped manifolds that appear to have the same volume. (You don't
    have to find all such pairs, though that can certainly be done.)

  \item For each pair, increase your confidence that the volumes are
    the same by using the \texttt{ManifoldHP} type, which works with
    quad-double precision.

    \item For each pair, try to either (a) find a common finite cover,
      or (b) show they are incommensurable by looking at the cusp
      density.  
  \end{enumerate}
\end{problems}


\end{document}