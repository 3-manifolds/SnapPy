\documentclass[tikz, a4paper]{nmd/hw}
\begin{document}
\classtemplate{Nathan PS}{Warwick School}{\hfill}

For further references and links: \url{http://dunfield.info/warwick2017}


\begin{problems}
\item Suppose $\cT$ is an ideal triangulation of a \3-manifold $M$ and
  $z \in \C^n$ satisfies Thurston's theorem and so gives a complete
  hyperbolic structure on $M$.  How would you use this information to
  find the holonomy representation
  $\rho \maps \pi_1(M) \to \PSL{2}{\C}$ of this hyperbolic structure? 
  In particular, you should have
  \[
    M = \leftquom{\Gamma}{\H^3}{4pt}{\big} \mtext{where $\Gamma = \rho\big(\pi_1(M)\big)$.}
  \]

  Hint: To get a presentation of $\pi_1(M)$, look at the 2-skeleton of
  the dual cellulation to $\cT$.  
  
\item Prove the theorem on page 4 of the notes for Lecture 4 for any
  $C^1$ function $f \maps \R \to \R$.  Hint: The Mean Value Theorem
  should be your friend.

\item SageMath and SnapPy are friends.
  \begin{enumerate}
  \item Find the first nontrivial knot in \texttt{HTLinkExteriors}
    which has the same Alexander polynomial as the unknot.

  \item Let $M$ be the knot exterior you found in part (a).  Via the
    verify module in SnapPy, use interval arithmetic to rigorously
    prove that this manifold is indeed hyperbolic.  Consequently, the
    corresponding knot is not the unknot.

  \item Find exact expressions for the tetrahedra shapes of
    $M$, which live in some number field.
    Hint: See \url{http://snappy.computop.org/snap.html}  

  \end{enumerate}

  
\end{problems}


\end{document}